% !TEX root = main.tex
\documentclass[
candidate, % document type
subf, % use and configure subfig package for nested figure numbering
times new roman % use Times New Roman font as main
]{disser}

% Кодировка и язык
\usepackage[T2A]{fontenc} % поддержка кириллицы
\usepackage[utf8]{inputenc} % кодировка исходного текста
\usepackage[english,russian]{babel} % переключение языков

% Геометрия страницы и графика
\usepackage[left=3cm, right=1cm, top=2cm, bottom=2cm]{geometry} % поля страницы
\usepackage{graphicx} % подключение графики
\usepackage{pdfpages} % вставка pdf-страниц

% Таблицы
\usepackage{array} % расширенные возможности для работы с таблицами
\usepackage{tabularx} % автоматический подбор ширины столбцов
\usepackage{dcolumn} % выравнивание чисел по разделителю

% Математика
\usepackage{bm} % полужирное начертание для математических символов
\usepackage{amsmath} % дополнительные математические возможности
\usepackage{amssymb} % дополнительные математические символы
\usepackage{mathrsfs}  % mathscr font

% Библиография и ссылки
\usepackage{cite} % поддержка цитирования
\usepackage[hidelinks]{hyperref} % создание гиперссылок

% bibliography
% \usepackage[
%     backend=biber
% ]{biblatex}
% \addbibresource{biblio/biblio.bib}

% Прочее
\usepackage{color} % работа с цветом
\usepackage{epstopdf} % конвертация eps в pdf
\usepackage{multirow} % объединение ячеек таблиц по вертикали
\usepackage{afterpage} % вставка материала после текущей страницы
\usepackage[font={normal}]{caption} % настройка подписей к рисункам и таблицам
\usepackage[onehalfspacing]{setspace} % полуторный интервал
\usepackage{fancyhdr} % установка колонтитулов
\usepackage{listings} % поддержка вставки исходного кода
\usepackage{float} % для позиционирования изображений

% for code listing
\usepackage{minted}
\usepackage{etoolbox}
\setminted{fontsize=\small,baselinestretch=0.5, linenos}
% \definecolor{codegreen}{rgb}{0,0.6,0}
% \definecolor{codegray}{rgb}{0.5,0.5,0.5}
% \definecolor{codepurple}{rgb}{0.58,0,0.82}
% \definecolor{backcolour}{rgb}{0.95,0.95,0.92}

% \lstdefinestyle{mystyle}{
%     backgroundcolor=\color{backcolour},   
%     commentstyle=\color{codegreen},
%     keywordstyle=\color{magenta},
%     numberstyle=\tiny\color{codegray},
%     stringstyle=\color{codepurple},
%     basicstyle=\ttfamily\footnotesize,
%     breakatwhitespace=false,         
%     breaklines=true,                 
%     captionpos=b,                    
%     keepspaces=true,                 
%     numbers=left,                    
%     numbersep=5pt,                  
%     showspaces=false,                
%     showstringspaces=false,
%     showtabs=false,                  
%     tabsize=2
% }

% \lstset{
%     style=mystyle,
%     extendedchars=true,
%     inputencoding=utf8
%     }

% dataframe to latex
\usepackage{{booktabs}}

% Установка шрифта Times New Roman
% \renewcommand{\rmdefault}{ftm}
% \usepackage{fontspec}
% \setmainfont{Times New Roman}

% Создание нового типа столбца для выравнивания содержимого по центру
\newcommand{\PreserveBackslash}[1]{\let\temp=\\#1\let\\=\temp}
\newcolumntype{C}[1]{>{\PreserveBackslash\centering}p{#1}}

% Настройка стиля страницы
\pagestyle{fancy}      % Использование стиля "fancy" для оформления страниц
\fancyhf{}              % Очистка текущих значений колонтитулов
\fancyfoot[C]{\thepage} % Установка номера страницы в нижнем колонтитуле по центру
\renewcommand{\headrulewidth}{0pt} % Удаление разделительной линии в верхнем колонтитуле

% Настройка подписей к изображениям и таблицам
\captionsetup{format=hang,labelsep=period}

% Использование полужирного начертания для векторов
\let\vec=\mathbf

% Установка глубины оглавления
\setcounter{tocdepth}{2}

% Указание папки для поиска изображений
\graphicspath{{images/}}

% Установка стилей страницы и главы
\pagestyle{footcenter}
\chapterpagestyle{footcenter}

% статистики работы
\usepackage{lastpage}
\usepackage[figure, table, chapter]{totalcount}

%сдвиг в списк
\usepackage{enumitem}
\setlist[enumerate]{topsep=0pt, parsep=-5pt, leftmargin=20mm}
\setlist[itemize]{topsep=0pt, parsep=-5pt, leftmargin=20mm}

% метаданные PDF
\hypersetup{
	pdfauthor={Полузёров Т. Д.},
	pdftitle={Модель индивидуальных рисков для портфеля однородных кредитов},
	pdfkeywords={СПИСОК КЛЮЧЕВЫХ СЛОВ, КАСАЮЩИХСЯ ТЕМЫ РАБОТЫ},
	pdfsubject={Модели, Риски},
	pdfcreator={xelatex},
	pdflang={Russian}}

% переопределение команд
\newcommand{\E}[1]{\mathbb{E}\left[#1\right]} % Мат ожидание
\newcommand{\D}[1]{\mathbb{D}\left[#1\right]} % Дисперсия
\newcommand{\COV}[2]{\textbf{Cov}\left(#1, #2\right)}
\newcommand{\Cov}{\textbf{Cov}}
\newcommand{\R}{\mathbb{R}}
\renewcommand{\epsilon}{\varepsilon}
\renewcommand{\phi}{\varphi}

\emergencystretch 3em

\begin{document}
\begin{titlepage}
    
    \begin{center}
        \textbf{БЕЛОРУССКИЙ ГОСУДАРСТВЕННЫЙ УНИВЕРСИТЕТ}
        
        \textbf{Факультет прикладной математики и информатики}
        
        \textbf{Кафедра теории вероятностей и математической статистики}
    \end{center}
    
    \vspace{180pt}
    
    \begin{center}
        Аннотация к магистерской диссертации
        
        \vspace{20pt}
        
        \textbf{Модели доходностей активов в средне-дисперсионном анализе Маркоцива на криптовалютных рынках}
        
        \vspace{20pt}
        
        Полузёров Тимофей Дмитриевич
        
        \vspace{20pt}
        
        Научный руководитель --- доктор физико-математических наук, профессор
        
        Харин Алексей Юрьевич
        
    \end{center}
    
    \vspace{200pt}
    
    \begin{center}
        Минск, 2025
    \end{center}
\end{titlepage}

\newpage

\begin{center}
    \textbf{РЕФЕРАТ}
\end{center}

    \textbf{Магистерская диссертация}, 43 страницы, 10 рисунков, 5 таблиц, 1 приложение.

    \textbf{Ключевые слова:}
    ПОРТФЕЛЬНАЯ ТЕОРИЯ, ИНВЕСТИЦИИ, АКТИВЫ, ВАЛЮТЫ, КРИПТОВАЛЮТЫ, СРЕДНЕ-ДИСПЕРИСИОННЫЙ АНАЛИЗ, ПРОГНОЗИРОВАНИЕ ВРЕМЕННЫХ РЯДОВ,
    ДОХОДНОСТЬ, АВТОРЕГРЕССИЯ, МАШИННОЕ ОБУЧНИЕ, БИРЖА.

    \textbf{Цель работы:}
    исследовать на реальных данных эффективность методов оценки средней ожидаемой доходности в портфельной теории Марковица.

    \textbf{Объект исследования:} 
    методы прогнозирования средней доходности, портфельная теория.

    \textbf{Предмет исследования:}
    эффективность методов оценки средней доходности и оценка доходностей соотвествующих портфелей.

    \textbf{Методы исследования:}
    методы теории вероятностей, математической статистики и временных рядов, методы регрессионного анализа, методы машинного обучения.

    \textbf{Результаты работы:}
    предложены методы оценки средних доходностей активов. На реальных данных исследованы доходности соотвествующих портфелей.
    Выполнена программная реализаци алгоритмов по определению оптимальных портфелей и оценка их доходностей.

    \textbf{Области применения:}
    фондовые, валютные, криптовалютные биржи. Инвестиционные проекты, страхование.

\newpage

\begin{center}
    \textbf{ABSTRACT}
\end{center}

\textbf{Master thesis}, 43 pages, 10 figures, 5 tables, 1 application.

\textbf{Keywords:}
PORTFOLIO THEORY, INVESTMENTS, ASSETS, CURRENCIES, CRYPTOCURRENCIES, MEAN-VARIANCE ANALYSIS, 
TIME SERIES FORECASTING, RETURN, AUTOREGRESSION, MACHINE LEARNING, STOCK EXCHANGE.

\textbf{The aim:}
to investigate the effectiveness of methods for estimating the average expected return in Markowitz's portfolio theory on real data.

\textbf{The object:}
methods for forecasting average returns, portfolio theory.

\textbf{Research methods:} 
methods of probability theory, mathematical statistics and time series, methods of regression analysis, methods of machine learning.

\textbf{The results:}
Methods for estimating average assets returns are proposed. 
The returns of the corresponding portfolios are studied using real data. 
A software implementation of algorithms for determining optimal portfolios and estimating their returns is completed.

\textbf{Application:}
stock, currency, cryptocurrency exchanges. Investment projects, insurance.

\end{document}