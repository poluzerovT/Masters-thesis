\newpage
\begin{center}
	\textbf{\large ЗАКЛЮЧЕНИЕ}
\end{center}
% \refstepcounter{chapter}
\addcontentsline{toc}{chapter}{ЗАКЛЮЧЕНИЕ}

В работе была рассмотрена проблема формирования оптимального портфеля с точки зрения ожидаемой доходности и принимаемого риска.
Был предложен метод оценки средних ожидаемых доходностей, необходимых для построения портфеля оптимального по Марковицу.
В рамках этого метода рассматривались некоторые модели прогнозирования временных рядов. 
В практической части были проверены на реальных данных доходности торговых стратетегий, построенных согласно теории Марковица,
где для прогнозирования неизвестных характеристик использовались рассмотренные модели.

По имеющимся эмпирическим результатам можно сделать следующие выводы:
\begin{itemize}
	\item активы имеют сильную положительную корреляцию
	\item стремление  сформировать портфель с большей доходностью влечет большие риски
	\item диверсификация действительно позволяет снижать риск портфеля
	% \item эмпирические фронтиры стратегий имеют выпуклую вверх форму, что согласуется с теорией
	\item как правило, формирование портфеля доминирует над инвестированием в отдельные активы
	\item линейная модель авторегрессии показала лучшее качество для оценки средней доходности
\end{itemize}

Полученные теоретические выкладки и программную реализацию формирования портфелей можно применять
при выборе своей инвестиционной стратегии. 

Дальнейшие шаги по исследованию данной темы могут быть следующими:
\begin{itemize}
	\item рассмотреть другие классы методов прогнозирования временных рядов
	\item помимо авторегрессионных признаков, учесть влияние внешних факторов на формирование цен
	\item расширить рассматриваемый набор активов
	\item исследовать другие таймфреймы и периоды инвестирования
\end{itemize}