\newpage
\begin{center}
	\textbf{\large 2. СРЕДНЕ-ДИСПЕРСИОННЫЙ АНАЛИЗ ПОРТФЕЛЯ}
\end{center}
\refstepcounter{chapter}
\addcontentsline{toc}{chapter}{. СРЕДНЕ-ДИСПЕРСИОННЫЙ АНАЛИЗ ПОРТФЕЛЯ}

\section{Основные понятия}

Будем рассматривать одношаговую задачу инвестирования.

Пусть инвестор имееот возможность разместить свой начальный капитал $x$ по акциям $A_1, \dots, A_N$, стоимость которых 
в момент $n=0$ равна соответственно
$S_0(A_1), \dots, S_0(A_N)$

Пусть $X_0(b) = b_1 S_0(A_1) + \dots + B_N S_0(A_N)$, где $b_i \ge 0$, $i=1, \dots N$. Иначе говоря, пусть
\begin{align}
b = (b_1, \dots, b_N )
\end{align}
есть портфель ценных бумаг, где $b_i$ -- число акций $A_i$ стоимостью $S_0(A_i)$.

Будем предпоалагать, что эволбция каждой акции $A_i$ определяется тем, что её цена $S_1(A_i)$ в момент $n=1$ подчиняется 
разностному уравнению
\begin{align}
\Delta S_1(A_i) = \rho(A_i) S_0(A_i)
\end{align}
или, что равносильно,
\begin{align}
S_1(A_i) = (1 = \rho(A_i))S_0(A_i)
\end{align}
где $\rho(A_i)$ --- случайная процентная ставка акции $A_i$, $\rho(A_i) > -1$.

Если инвестор выбрал портфель $b = (b_1, \dots, b_N )$, то его начальный капитал $X_0(b) = x$ превратится в 
\begin{align}
X_1(b) = b_1 S_1(A_1) + \dots + b_N S_1(A_N),
\end{align}
и эту величину желательно сделать <<побольше>>. Это желание, однако, должно рассматриваться с учетом <<риска>>, 
связанного с получением <<большего>> дохода.

С этой целью Г. Марковитц рассматривает две характеристики капитала $X_1(B)$:
\begin{align}
\E{X_1(b)}
\end{align} -- математическое ожидание
и
\begin{align}
\D{X_1(b)}
\end{align} -- дисперсию.

Имея эти две характеристики, можно по-разному формулировать оптимизационную задачу выбора наилучшего портфеля в
зависимости от критерия оптимальности.

Можно, например, задаться вопросом о том, на каком портфеле $b^*$ достигается максимум некоторой целевой функции 
$f = f(\E{X_1(b)}, \D{X_1(b)})$ при <<бюджетном ограничении>> на класс допустимых портфелей:
\begin{align}
B(x) = \{b=(b_1, \dots, b_N): b_i \ge 0, X_0(b) = x\}, x > 0
\end{align}

Естественна и следующая вариационная постановка: найти
\begin{align}
\inf \D{X_1(b)}
\end{align}
в предположении, что $\inf$ берется по тем портфелям $b$, для которых выполнены ограничения
\begin{align}
b \in B(x),
\end{align}
\begin{align}
\E{X_1(b)} = m,
\end{align}
где $m$ -- некоторая константа.

\section{Сведение к процентным ставкам}

Покажем теперь,что в одношаговой задаче оптимизации портфеля ценных бумаг можно вместо величин $(s_1(A_1), \dots, S_1(A_N))$
работать непосредственно с процентными ставками $(\rho(A_1), \dots, \rho(A_N))$, подразумевая под этим следующее.

Пусть $b \in B(X)$, т.е. $x = b_1 S_0(A_1) + \dots + b_N S_0(A_N)$. Введем величины $d = (d_1, \dots, d_N)$, полагая
\begin{align}
d_i = \frac{b_i S_0(A_i)}{x}
\end{align}.

Поскольку $b \in B(X)$, получаем, что $d_i \ge 0$ и $\sum_{i=1}^{N} = 1$. Представим капитал $X_1(B)$ в виде
\begin{align}
X_1(b) = (1 + R(b))X_0(b) ,
\end{align}
и пусть
\begin{align}
\rho(d) = d_1 \rho(A_1) + \dots + d_N \rho(A_N) .
\end{align}

Ясно, что 
\begin{align}
R(b) = 
	\frac{X_1(b)}{X_0(b)} - 1 = 
	\frac{X_1(b)}{x} - 1 = \\
	\frac{\sum b_i S_1(A_i)}{x} - 1 =
	\sum d_i \frac{S_1(A_i)}{S_0(A_i)} - 1 = \\
	\sum d_i \left( \frac{S_1(A_i)}{S_0(A_i)} - 1\right) =
	\sum d_i \rho(A_i) = 
	\rho(d)
\end{align}

Итак,
\begin{align}
R(b) = \rho(d),
\end{align}
откуда следует, что если $d = (d_1, \dots, d_N)$ и $b = (b_1, \dots, b_N)$ связаны соотношениями 
$d_i = \frac{b_i S_0(A_i)}{x}, i=1, \dots, N$, то для $b \in B(x)$ выполняется равенство
\begin{align}
X_1(b) = x(1 + \rho(d)),
\end{align}
и, следовательно, с точки зрения оптимизационных задач для $X_1(b)$ можно оперировать с соотвествующими задачами
для $\rho(d)$.

\section{Диверсификация портфеля}

Обратимся теперь к вопросу о том, как диверсификацией можно добится сколь угодно малого (несистематического) риска, измеряемого дисперсией
или стандарным отклонением величин $X_1(b)$.

С этой целью рассмотрим для начала пару случайных величин $\xi_1$ и $\xi_2$ с конечными вторыми моментами. Тогда если $c_1$ и $c_2$ -- константы,
$\sigma_i = \sqrt{\D{\xi_i}}, i=1,2$, то 
\begin{align}
\D{c_1 \xi_1 + c_2 \xi_2} = 
	(c_1 \sigma_1 - c_2 \sigma_2)^2 + 2 c_1 c_2 \sigma_1 \sigma_2 (1 + \sigma_{12}),
\end{align}
где $\sigma_{12} = \frac{\COV{\xi_1}{\xi_2}}{\sigma_1 \sigma_2}, \COV{\xi_1}{\xi_2} = \E{\xi_1 \xi_2} - \E{\xi_1} \cdot \E{\xi_2}$.
Отсюда ясно, что если $c_1 \sigma_1 = c_2 \sigma_2$ и $\sigma_{12} = -1$, то $\D{c_1 \xi_1 + c_2 \xi_2} = 0$.
ак
Таким образом, если величины $\xi_1$ и $\xi_2$ отрицательно коррелированы с коэффициентом корреляции $\sigma_{12} = -1$, то таким подбором
констант $c_1$ и $c_2$, что $c_1 \sigma_1 = c_2 \sigma_2$, получаем комбинацию $c_1 \xi_1 + c_2 \xi_2$ с нулевой дисперсией. Но,
конечно, при этом среднее значение $\E{c_1 \xi_1 + c_2 \xi_2}$ может оказаться достаточно малым. (Случай $c_1 = c_2 = 0$ для задачи 
оптимизации не интересен в силу условия $b \in B(X))$.

Из этих элементарных рассуждений ясно, что при заданных ограничениях на $(c_1, c_2)$ и класс величин $(\xi_1, \xi_2)$ при решении задачи о том,
чтобы сделать $\E{c_1 \xi_1 + c_2 \xi_2}$ <<побольше>>, а $\D{c_1 \xi_1 + c_2 \xi_2}$ <<поменьше>>, надо стремиться к выбору таких пар 
$(\xi_1, \xi_2)$, для которыз их ковариация была бы как можно ближе к минус единице.

Изложенный эффект отрицательной коррелированности, называемый эффектом Марковитца, является одной из основных идей диверсификации при инвестировании ---
при составлении портфеля ценных бумаг надо стремиться к тому, чтобы вложения делались в бумаги, среди которых по возможности много отрицательно коррелированных.

Другая идея, лежащая в основе диверсификации, основана на следующем соображении.

Пусть $\xi_1, \dots, \xi_N$ --- последоватльность некоррелированных случайных величин с дисперсиями $\D{\xi_i} \le C, i=1, \dots, N$, 
где $C$ --- некоторая константа. Тогда
\begin{align}
\D{d_1 \xi_1 + \dots + d_N \xi_N} = \sum_{i=1}^{N} d_i^2 \D{\xi_i} \le C \sum_{i=1}^{N} d_i^2 .
\end{align}

Поэтому, взяв, например, $d_i = \frac{1}{N}$, находим, что
\begin{align}
\D{d_1 \xi_1 + \dots + d_N \xi_N} \le \frac{C}{N} \rightarrow 0, N \rightarrow \infty
\end{align}

Этот эффект некоррелиованности говорит о том, что если инвестирование производится в некоррелированные ценные бумаги, то для уменьшения риска,
т. е. дисперсии $\D{d_1 \xi_1 + \dots + d_N \xi_N}$, надо по возможности брать их число $N$ как можно большим.

Вернемся к вопросу о дисперсии $\D{\rho(d)}$ величины
\begin{align}
\rho(d) = d_1 \rho(A_1) + \dots + d_N \rho(A_N) .
\end{align}

Имеем
\begin{align}
\D{\rho(d)} = \sum_{i=1}^{N} d_i^2 \D{\rho(A_i)} + \sum_{i, j = 1, i \ne j}^{N} d_i d_j \COV{\rho(A_i)}{\rho(A_j)} .
\end{align}

Возьмем здесь $d_i = \frac{1}{N}$. Тогда
\begin{align}
\sum_{i=1}^{N} d_i^2 \D{\rho(A_i)} = 
	\left(\frac{1}{N}\right) \cdot N \cdot \frac{1}{N} \sum_{i=1}^{N} \D{\rho(a_i)} =
	\frac{1}{N} \cdot \overline{\sigma}_N^2 ,
\end{align}
где $\bar{\sigma}_N^2 = \frac{1}{N} \sum_{i=1}^{N} \D{\rho(A_i)}$ --- средняя дисперсия. Далее,
\begin{align}
\sum_{i, j = 1, i \ne j}^{N} d_i d_j \COV{\rho(A_i)}{\rho(A_j)} =
	\left(\frac{1}{N}\right)^2 N (N - 1) \overline{\Cov}_N ,
\end{align}
где $\overline{\Cov}_N$ есть средняя ковариация
\begin{align}
\overline{\Cov}_N = \frac{1}{N (N - 1)} \sum_{i, j = 1, i \ne j}^{N} \COV{\rho(A_i)}{\rho(A_j)} .
\end{align}

Таким образом,
\begin{align}
\D{\rho(d)} = \frac{1}{N} \overline{\sigma}_N^2 + \left(1 - \frac{1}{N}\right) \overline{\Cov}_N ,
\end{align}
и ясно, что если $\overline{\sigma}_N^2 \le C$ и $\overline{\Cov}_N \rightarrow \overline{\Cov}$ при $N \rightarrow \infty$, то
\begin{align}
\D{\rho(d)} \rightarrow \overline{\Cov}, N \rightarrow \infty .
\end{align}

Из этой формулы мы видим, что если $\overline{\Cov}$ равна нулю, то диверсификацией с достаточно большим $N$ риск инвестирования,
т.е. $\D{\rho(d)}$, может быть сделан сколь угодно малым. К сожалению, если рассматривать, скажем, рынок акций, то на нем, как правило,
имеется положительная корреляция в ценах (они движутся довольно-таки согласованно в одном направлении), что приводит к тому, что
$\overline{\Cov}_N$ не стремится к нулю при $N \rightarrow \infty$. Предельное значение $\overline{\Cov}$ и есть тот систематический,
иначе --- рыночный --- риск, который присущ рассматриваемому рынку и диверсификацией не может быть редуцирован. Первый же член в формуле
!!! определяет несистематический риск, который может быть редуцирован, как мы видели, выбором большого числа акций.