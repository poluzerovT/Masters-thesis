\newpage
\begin{center}
	\textbf{\large 3. ПРОВЕРКА СТРАТЕГИЙ НА РЫНОЧНЫХ ДАННЫХ}
\end{center}
\refstepcounter{chapter}
\addcontentsline{toc}{chapter}{3. ПРОВЕРКА СТРАТЕГИЙ НА РЫНОЧНЫХ ДАННЫХ}

\section{Подготовка данных}

Выгрузим цены закрытия дневных свечей.

\section{Оценка ожидаемой доходности}

Рассмотрим несколько подходов к оценке доходности.

1. На основании исторических данных --- предполагается эргодичность?

2. Прогнозирование временных рядов

3. Последнее значение --- предположение мартингальности

\section{Оценка ковариации доходностей}

Оценим ковариации по историческим данным

\section{Множество оптимальных портфелей}

Для каждого способа оценки доходностей построим фронтиру оптимальных портфелей

\section{Проверка стратегий на исторических данных}

Используя кросс-проверку протестируем стратегии

