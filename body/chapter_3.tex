\newpage
\begin{center}
	\textbf{\large 3. ПРОВЕРКА СТРАТЕГИЙ НА РЫНОЧНЫХ ДАННЫХ}
\end{center}
\refstepcounter{chapter}
\addcontentsline{toc}{chapter}{3. ПРОВЕРКА СТРАТЕГИЙ НА РЫНОЧНЫХ ДАННЫХ}

В этой главе оценивается качество моделей для прогнозирования ожидаемых средних доходностей
и оцениваются доходности соотвествующих торговых стратегий.
Под стратегией понимается построенный на определенный момент времени и с определенным сроком инвестирования
портфель. Рассматриваются стратегии двух типов. Тривиальные, когда капитал инвестируется в один отдельный актив 
или в весь рынок одновременно, и стратегии, основанные на теории Марковица, где для оценки средней ожидаемой 
доходности используется некоторая модель из главы 2.


\section{Подготовка данных}

Рыночные данные были выгружены с помощью API с криптовалютной биржи OKX \cite{okx}.
В качестве доступных для торговли активов рассматриваются 8 наиболее популярных крюптовалют.
Временной период с 1 января 2022 по  1 января 2025.
Был выбран дневной таймфрейм.
Период инвестирования 1 неделя.

На графике \ref{fig:prices} изображены динамики цен активов.

Перейдем от цен к недельным доходностям. Временные ряды, соответсвующие доходностям,
представлены на графике \ref{fig:returns}, а некоторые статистики относительно распределений
доходностей в таблице \ref{tab:returns_describe}

\begin{table}[h]
\caption{Доходности активов}
\begin{tabular}{lrrrrrrrr}
\toprule
 & BTC & ETH & DOT & OKB & XRP & SOL & TRX & LTC \\
\midrule
mean & 0.0073 & 0.0038 & -0.0031 & 0.0077 & 0.0132 & 0.0114 & 0.0105 & 0.0025 \\
std & 0.0779 & 0.0961 & 0.1110 & 0.0936 & 0.1328 & 0.1482 & 0.0800 & 0.1001 \\
min & -0.3328 & -0.3830 & -0.3925 & -0.3591 & -0.3596 & -0.6018 & -0.3162 & -0.3392 \\
25\% & -0.0358 & -0.0477 & -0.0724 & -0.0401 & -0.0506 & -0.0765 & -0.0236 & -0.0513 \\
50\% & 0.0026 & -0.0013 & -0.0084 & -0.0016 & -0.0003 & -0.0034 & 0.0106 & 0.0001 \\
75\% & 0.0446 & 0.0555 & 0.0573 & 0.0494 & 0.0430 & 0.0906 & 0.0373 & 0.0546 \\
max & 0.3566 & 0.5056 & 0.6188 & 0.4003 & 1.0235 & 0.7409 & 0.7407 & 0.5294 \\
\bottomrule
\end{tabular}
\label{tab:returns_describe}
\end{table}


\begin{figure}[H]
	\centering
	\includegraphics[width=\textwidth]{prices.png}
	\caption{Цены активов}
	\label{fig:prices}
\end{figure}

\begin{figure}[H]
	\centering
	\includegraphics[width=\textwidth]{returns.png}
	\caption{Доходности активов}
	\label{fig:returns}
\end{figure}
Можно видеть редкие но достаточно сильные скачки.

Распределение доходностей активов представлено на гистограммах на рисунке \ref{fig:rois_hist}

\begin{figure}[H]
	\centering
	\includegraphics[width=\textwidth]{rois_hist.png}
	\caption{Гистограммы доходностей активов}
	\label{fig:rois_hist}
\end{figure}

Из гистограмм видно, что распределение доходностей унимодально и имеет тяжелые хвосты.

На графике \ref{fig:rois_mean_std} сравниваются активы с точки зрения 
среднего и стандартного отклонения доходности.

\begin{figure}[H]
	\centering
	\includegraphics[width=\textwidth]{rois_mean_std.png}
	\caption{Среднее и стандартное отклонение доходностей}
	\label{fig:rois_mean_std}
\end{figure}

Разделим имеющиеся данные на валидационную и тестовую выборки. В качестве тестовых данных возьмем 2024 год.
По валидационной выборке подберем оптимальное число лагов ряда и индивидуальные гиперпараметры алгоритмов.

В дальнейшем тестовые данные будут использоваться для:
\begin{enumerate}
	\item оценки качества прогнозирования средней ожидаемой доходности
	\item тестирования портфельных стратегий
\end{enumerate}

Из тестовых днных формируется набор тест-кейсов на которых и оценивается качество.
Процесс формирования тест-кейсов схематично проилюстрирован на рисунке \ref{fig:ts_csv}.
\begin{figure}[H]
	\centering
	\includegraphics[width=\textwidth]{images/ts_cv}
	\caption{Формирование тест-кейсов из тестовых данных}
	\label{fig:ts_csv}
\end{figure}

Следующим этапом идет расчет необходимых параметров для оптимизации портфеля --- 
оценка ковариаций и прогноз средних значений доходности.

\section{Оценка ковариации между активами}

Особую сложность предстваляет задача прогноза будущей ковариации временных рядов. 
Вполне естественным вляется предположение стационарности ковариации во времени. 
Поэтому воспользуемся выборочной оценкой ковариации по историческим данным.

Имея $r_t$ - вектор-столбец доходностей в момент времени t, по истории наблюдений $r_1, \cdots r_n$ 
выборочная ковариация $\hat{\Sigma}$ рассчитывается как 
\begin{align}
	\hat{\Sigma} = \frac{1}{n} \sum_{t=1}^{n}(r_t - \overline{r}) \cdot (r_t - \overline{r})^T
\end{align}
где $\overline{r} = \frac{1}{n} \sum_{t=1}^{n} r_t$.

Корреляция Пирсона между доступными активами представлена на рисунке \ref{fig:corr}.

\begin{figure}[H]
	\centering
	\includegraphics[width=\textwidth]{corr.png}
	\caption{Корреляции доходностей активов}
	\label{fig:corr}
\end{figure}

Активы имею сильную положительную корреляцию. Согласно теории, описанной в главе 1, желательным является
возможность инвестирования в активы с отрицательной корреляцией. Однако, это не помешает редуцировать риски портфеля.

Для формирования портфеля остается оценить средние ожидаемые доходности. Перейдем к рассмотрению моделей для
прогнозирования этих значений.

\section{Модели оценки средней доходности}

Задача оценки средней ожидаемой доходности сводиться к умению прогнозировать значение основываясь на стории наблюдений.
Для этого подходят классические статистические модели, модел имашинного обучения и нейросети, адаптированные для прогнозирования 
временных рядов. Реалиации моделей взяты из программных библиотек для Python \cite{skforecast} и \cite{sklearn}.

Ограничимся рассмотрением следующих моделей:
\begin{enumerate}
	\item NAIVE - прогнозирование историческим средним
	\item MARTINGAL - прогноз последним наблюдаемым значением
	\item ARIMA - модель авторегрессии и скользящего среднего (см. \ref{sec:ARIMA})
	\item LR - линейная регрессия (см. \ref{sec:LR})
	\item RF - случайный лес (см. \ref{sec:RF})
\end{enumerate}

Для каждого актива будем строить отдельную модель не принимающую в расчет историю других активов.
Таким образом, для прогноза будующих доходностей активов необходимо построить моделей по числу активов.

Некоторые модели (ARIMA, RF) --- допускают свободу в выборе гиперпараметров. 
Подбор гиперпараметров моделей осуществлялся по тренировочной выборке.

Качество прогнозирования моделей оценивается с помощью среднеквадратичной ошибки MSE (Mean Squared Error):
\begin{align}
	MSE = \frac{1}{n} \sum_{i=1}^{n} (r_i - \hat{r}_i)^2
\end{align}
где $r_i$ - истинное значение доходности,
а $\hat{r}_i$ - прогнозное значение модели на $i$-м объекте тестовой выборки.

Результаты оценки качества прогнозирования на тестовых данных представлены в таблице \ref{tab:ml_eval_metrics}

\begin{table}[h]
\caption{Качество прогнозирования MSE$\cdot 10^4$}
\label{tab:ml_eval_metrics}
\begin{tabular}{lrrrrr}
\toprule
 & NAIVE & MARTINGAL & LR & ARIMA & RF \\
\midrule
BTC & 5.63 & 1.20 & 1.58 & 1.62 & 2.07 \\
ETH & 8.00 & 1.99 & 4.47 & 3.70 & 5.05 \\
DOT & 16.51 & 3.89 & 4.09 & 3.98 & 5.28 \\
OKB & 6.06 & 1.56 & 1.77 & 1.95 & 2.05 \\
XRP & 24.33 & 5.04 & 6.98 & 5.71 & 6.53 \\
SOL & 21.19 & 4.47 & 11.86 & 6.16 & 5.31 \\
TRX & 7.96 & 1.85 & 4.19 & 5.30 & 6.04 \\
LTC & 8.53 & 2.66 & 2.24 & 3.22 & 5.11 \\
\bottomrule
\end{tabular}
\end{table}


Наихудшее значениие показывает подход NAIVE.
Это обусловлено резким ростом цен в 2024 году после относительно спокойной динамики.
Метод MARTINGAL показывает хорошие результаты в случае рядов с затяжным трендов.
Остальные модели показывают сопоставимое качество.

\section{Оценка доходностей стратегий}

Будем рассматривать стратегии двух видов:
\begin{itemize}
	\item тривиальные 
	\item основанные на идеи Марковица
\end{itemize}

Среди тривиальных стратегий выберем следующие:
\begin{enumerate}
	\item UNIFORM - равномерное инвестированиие во все доступные активы
	\item MOST RISKY - актив с наибольней дисперсией доходности
	\item LESS RISKY - актив с наименьшей дисперсией доходности
	\item BEST RETURN - актив с наибольшей средней доходностью
	\item WORST RETURN - актив с наименьшей средней доходностью
\end{enumerate}

Стратегии Марковица определяются риск-параметром и моделью оценки средней ожидаемой доходностью.
Риск-параметр будем воспринимать как параметризацию класса стратегий с определенной моделью оценки средней ожидаемой доходности.
Таким образом, одной стратегии Марковица соответсвует множество стратегий с разным риск-параметром. 
Это множество стратегий будет называть фронтирой.

На каждом тест-кейсе с помощью стратегии формируется инвестиционный портфель
в расчете на единичную сумму инвестирования и оценивается доходность ROI (Return On Investment) полученной стратегии.
Метрика ROI определяется для стратегии аналогично доходности портфеля \ref{eq:ROI}.

На графике \ref{fig:result_frontier} представлены фронтиры соответсвующие торговым стратегиям.
Серым цветом отмечены тривиальные портфели.
По оси абсцисс отложены стандартые отклонения ROI, а по оси ординат --- средние значение ROI.

\begin{figure}[H]
	\centering
	\includegraphics[width=\textwidth]{result_frontiers.png}
	\caption{Результаты тестирования стратегий}
	\label{fig:result_frontier}
\end{figure}

Более детально средние значения и стандартные отклонения ROI стратегий представлены в таблицах
\ref{tab:roi_mean} и \ref{tab:roi_std} соответственно.

Метрики тривиальных портфелей представлены в таблице \ref{tab:trivial_rois}

\begin{table}[h]
\caption{Средние ROI $\cdot 10^3$}
\label{tab:roi_mean}
\begin{tabular}{lrrrrr}
\toprule
 &  0.01 &  0.26 &  0.51 &  0.75 &  1.00 \\
\midrule
NAIVE & 6.9451 & 8.0025 & 11.0462 & 10.6657 & 15.4227 \\
MARTINGAL & 2.9859 & 6.0019 & 3.6178 & 6.5656 & 3.9942 \\
LR & 8.2600 & 19.0185 & 21.5537 & 22.7442 & 23.1668 \\
ARIMA & 7.4648 & 14.7066 & 13.9296 & 13.1520 & 15.9971 \\
RF & 5.2633 & 9.4236 & 7.2398 & 4.3777 & 4.5050 \\
\bottomrule
\end{tabular}
\end{table}


\begin{table}[H]
\caption{Стандартное отклонение ROI $\cdot 10^2$}
\label{tab:roi_std}
\begin{tabular}{lrrrrr}
\toprule
 &  0.01 &  0.25 &  0.50 &  0.75 &  1.00 \\
\midrule
NAIVE & 3.4328 & 5.3375 & 8.3427 & 9.8131 & 10.6650 \\
MARTINGAL & 3.8577 & 9.4886 & 9.8051 & 11.0111 & 10.7128 \\
LR & 3.7892 & 10.6314 & 13.1000 & 14.3570 & 14.3071 \\
ARIMA & 3.5454 & 10.0037 & 11.5506 & 11.1001 & 12.4518 \\
RF & 3.9215 & 10.7564 & 11.3710 & 11.9397 & 11.9118 \\
\bottomrule
\end{tabular}
\end{table}


\begin{table}[h]
\caption{Тривиальные портфели}
\label{tab:trivial_rois}
\begin{tabular}{lrr}
\toprule
 & mean ROI $\cdot 10^3$ & std ROI $\cdot 10^2$ \\
\midrule
UNIFORM & 15.5372 & 8.1775 \\
MOST RISKY & 18.5904 & 11.3970 \\
LESS RISKY & 21.1635 & 9.6881 \\
BEST RETURN & 1.2790 & 7.7329 \\
WORST RETURN & 4.9217 & 12.6324 \\
\bottomrule
\end{tabular}
\end{table}


На тестовых данных метод оценки средней доходности с помощью модели линейной регрессии строго доминирует над всеми остальными методами
при любом значении риск-параметра. 

Модели ARIMA и NAIVE позволяют контролировать риски портфеля за счет изменения риск-параметра.
При формировании портфеля минимального риска эти модели сопоставимы между собой.

Модели RF и MARTINGAL показали неудовлетворительное качество. При увеличении толерантности к риску, доходность портфеля не возрастает,
что противоречит теории и здравому смыслу.

Инвестирование равных долей во все доступные активы является примелимым, однако такой подход не позволяет контролировать риски.

Формирование портфеля состоящего только из одного актива сильно непредсказуемо и, следовательно, рисковано 
из-за сильных колебаний цен (специфика криптовалютных рынков).
