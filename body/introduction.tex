\newpage
\addcontentsline{toc}{chapter}{ВВЕДЕНИЕ}
%\onehalfspacing
\begin{center}
	\textbf{\large ВВЕДЕНИЕ}
\end{center}

Портфельная теория Марковица дает точный ответ на вопрос выбора портфеля если известны математическое ожидание
и ковариации случайных величин, описывающих доходности ативов за будущий период инвестирования. На практике конечно эти значения неизвестны
и приходится их оценивать (прогнозировать). 
Были разработаны модели оценки характеристик будущих доходностей, которые помимо этого дают ответ на вопрос как они зависят
от <<большого рынка>> в целом. 

Модель CAPM (Capital Asset Pricing Model), разработанная У.Шарпом и Дж.Линтером \cite{sharp_capm}, \cite{linter_capm},которая базируется на
концепции равновесного рынка имоделирует линейную зависимость между доходностями активов и <<большим рынком>>.

Более современная теория --- теория APT(Arbitrage Pricing Theory) С. Росса и Р.Ролла \cite{ross_apt}, \cite{roll_apt}, исходящая из
многофакторной модели зависимости доходности активовот некоторых факторов (необязательно рыночных). 
Эта теория опирается на концепцию отсутсвия асимптотического арбитража.

К задаче оценки характеристик доходностей можно подойти иначе. Имея в распоряжении историю наблюдений за ценами активов можно поставить задачу
спрогнозировать будущие цены на основе предыдущей динамики. Для решения этой задачи можно применить современные методы
прогнозирования временных рядов. К таким методам относятся классические статистические модели, модели машинного обучения, модели временных рядов.

Идеи портфельной теории можно применить не только к выбору портфеля акций. Например, когда отдельный человек желая сохранить свой капитал,
принимает решение о том как разместить деньги по депозитам, купить валюту, страховку и так далее. 
А учитывася растущую популярность криптовалют в последнее время, особенный интерес представляет применение теории в этой области.

Криптовалюта --- это альтернативный вид валюты в цифровой или виртуальной форме, для защиты транзакций используется криптография. 
В 2009 года был создан первый криптовалютный токен --- Bitcoin. Принципы его работы были описаны в статье \cite{satoshi}. Далее выпускались
и дргуе токены, напрмер Etherium, Litecoin.
Торговля токенами ведется в интернете на специализированных биржах. Криптовалютные биржи устроены по тем же принципам что и фондовые и валютные биржри.
Поведение цен на криптовалюты структурно отличается от поведения цен на обыкновеннные акции. Цены криптовалют склонны к очень резким скачкам и как
правило в них отсутсвует долгосрочный тренд. В целом динамика цен более волатильная, причем волатильность не постонянна во времени.
Цифровизация торговли позвляет иметь моентальный доступ к котиовкам и вести активную торговлю. 
Это дает возможность трейдерам заниматься спекуляциями, а инвесторам --- возможность среднесрочного и долгосрочного инвестирования.

В этой работе усовершествуется подхода Марковица путем использования продвинутых методов прогнозирования временных рядов для оценки будущих доходностей.
Теоретические модели адаптируются для формирования однопериодных торговых стратегий на криптовалютном рынке. Доходность полученных стратегий проверяется
на реальных данных. Также рассматривается вопрос диверсификации портфеля и контроля риска.