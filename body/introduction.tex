\newpage
\addcontentsline{toc}{chapter}{ВВЕДЕНИЕ}
%\onehalfspacing
\begin{center}
	\textbf{\large ВВЕДЕНИЕ}
\end{center}

Портфельная теория Марковица дает точный ответ на вопрос выбора портфеля если известны математическое ожидание
и дисперсия случайных величин, описывающих доходности ативов за будущий период инвестирования. На практике конечно эти значения неизвестны
и приходится их оценивать (прогнозировать). Имея в распоряжении историю наблюдений за ценами активов можно поставить задачу
спрогнозировать будущие цены на основе предыдущей динамики. Для решения этой задачи можно применить современные методы
прогнозирования временных рядов. К таким методам относятся классические статистические модели, модели машинного обучения, модели временных рядов.

Идеи портфельной теории можно применить не только к выбору портфеля акций. Например, когда отдельный человек желая сохранить свой капитал,
принимает решение о том как разместить деньги по депозитам, купить валюту, страховку и так далее.

Большую популярность в последнее время получили криптовалюты. Торговля ведется на специальных интернет биржах.
Полученные теоретические подходы к формированию портфеля будут проверены на реальных данных.

Эта работа предследует несколько целей:
\begin{enumerate}
	\item проверить, позволяет ли диверсификация снижать риск портфеля
	\item сравнить модели оценки ожидаемой доходности
	\item оценить доходность портфелей
\end{enumerate}