\newpage
\begin{center}
	\textbf{\large 1. РОЛЬ ДАЛЬНОДЕЙСТВИЯ ПРИТЯЖЕНИЯ В ПРОСТЫХ ЖИДКОСТЯХ}
\end{center}
\refstepcounter{chapter}
\addcontentsline{toc}{chapter}{1. РОЛЬ ДАЛЬНОДЕЙСТВИЯ ПРИТЯЖЕНИЯ В ПРОСТЫХ ЖИДКОСТЯХ}

\section{Основные понятия}

Будем рассматривать одношаговую задачу инвестирования.

Пусть инвестор имееот возможность разместить свой начальный капитал $x$ по акциям $A_1, \dots, A_N$, стоимость которых 
в момент $n=0$ равна соответственно
$S_0(A_1), \dots, S_0(A_N)$

Пусть $X_0(b) = b_1 S_0(A_1) + \dots + B_N S_0(A_N)$, где $b_i \ge 0$, $i=1, \dots N$. Иначе говоря, пусть
\[
b = (b_1, \dots, b_N )
\]
есть портфель ценных бумаг, где $b_i$ -- число акций $A_i$ стоимостью $S_0(A_i)$.

Будем предпоалагать, что эволбция каждой акции $A_i$ определяется тем, что её цена $S_1(A_i)$ в момент $n=1$ подчиняется 
разностному уравнению
\[
\Delta S_1(A_i) = \rho(A_i) S_0(A_i)
\]
или, что равносильно,
\[
S_1(A_i) = (1 = \rho(A_i))S_0(A_i)
\]
где $\rho(A_i)$ -- случайная процентная ставка акции $A_i$, $\rho(A_i) > -1$.

Если инвестор выбрал портфель $b = (b_1, \dots, b_N )$, то его начальный капитал $X_0(b) = x$ превратится в 
\[
X_1(b) = b_1 S_1(A_1) + \dots + b_N S_1(A_N),
\]
и эту величину желательно сделать <<побольше>>. Это желание, однако, должно рассматриваться с учетом <<риска>>, 
связанного с получением <<большего>> дохода.

С этой целью Г. Марковитц рассматривает две характеристики капитала $X_1(B)$:
\[
\E{X_1(b)}
\] -- математическое ожидание
и
\[
\D{X_1(b)}
\] -- дисперсию.

Имея эти две характеристики, можно по-разному формулировать оптимизационную задачу выбора наилучшего портфеля в
зависимости от критерия оптимальности.

Можно, например, задаться вопросом о том, на каком портфеле $b^*$ достигается максимум некоторой целевой функции 
$f = f(\E{X_1(b)}, \D{X_1(b)})$ при <<бюджетном ограничении>> на класс допустимых портфелей:
\[
B(x) = \{b=(b_1, \dots, b_N): b_i \ge 0, X_0(b) = x\}, x > 0
\]

Естественна и следующая вариационная постановка: найти
\[
\inf \D{X_1(b)}
\]
в предположении, что $\inf$ берется по тем портфелям $b$, для которых выполнены ограничения
\[
b \in B(x),
\]
\[
\E{X_1(b)} = m,
\]
где $m$ -- некоторая константа.

\section{Сведение к процентным ставкам}

Покажем теперь,что в одношаговой задаче оптимизации портфеля ценных бумаг можно вместо величин $(s_1(A_1), \dots, S_1(A_N))$
работать непосредственно с процентными ставками $(\rho(A_1), \dots, \rho(A_N))$, подразумевая под этим следующее.

Пусть $b \in B(X)$, т.е. $x = b_1 S_0(A_1) + \dots + b_N S_0(A_N)$. Введем величины $d = (d_1, \dots, d_N)$, полагая
\[
d_i = \frac{b_i S_0(A_i)}{x}
\].

Поскольку $b \in B(X)$, получаем, что $d_i \ge 0$ и $\sum_{i=1}^{N} = 1$. Представим капитал $X_1(B)$ в виде
\[
X_1(b) = (1 + R(b))X_0(b) ,
\]
и пусть
\[
\rho(d) = d_1 \rho(A_1) + \dots + d_N \rho(A_N) .
\]

Ясно, что 
\[
R(b) = 
	\frac{X_1(b)}{X_0(b)} - 1 = 
	\frac{X_1(b)}{x} - 1 =
	\frac{\sum b_i S_1(A_i)}{x} - 1 =
	\sum d_i \frac{S_1(A_i)}{S_0(A_i)} - 1 = 
	\sum d_i \left( \frac{S_1(A_i)}{S_0(A_i)} - 1\right) =
	\sum d_i \rho(A_i) = 
	\rho(d)
\]

Итак,
\[
R(b) = \rho(d),
\]
откуда следует, что если $d = (d_1, \dots, d_N)$ и $b = (b_1, \dots, b_N)$ связаны соотношениями 
$d_i = \frac{b_i S_0(A_i)}{x}, i=1, \dots, N$, то для $b \in B(x)$ выполняется равенство
\[
X_1(b) = x(1 + \rho(d)),
\]
и, следовательно, с точки зрения оптимизационных задач для $X_1(b)$ можно оперировать с соотвествующими задачами
для $\rho(d)$.

