% RUSSIAN
\newpage
\addcontentsline{toc}{chapter}{ОБЩАЯ ХАРАКТЕРИСТИКА РАБОТЫ}


\begin{center}
	\textbf{\large ОБЩАЯ ХАРАКТЕРИСТИКА РАБОТЫ}
\end{center}


\textbf{Ключевые слова:}
ПОРТФЕЛЬНАЯ ТЕОРИЯ, ИНВЕСТИЦИИ, АКТИВЫ, ВАЛЮТЫ, КРИПТОВАЛЮТЫ, СРЕДНЕ-ДИСПЕРИСИОННЫЙ АНАЛИЗ, ПРОГНОЗИРОВАНИЕ ВРЕМЕННЫХ РЯДОВ,
ДОХОДНОСТЬ, АВТОРЕГРЕССИЯ, МАШИННОЕ ОБУЧНИЕ, БИРЖА.

\textbf{Цель работы:} 
исследовать на реальных данных эффективность методов оценки средней ожидаемой доходности в портфельной теории Марковица.

\textbf{Объект исследования:} 
методы прогнозирования средней доходности, портфельная теория.

\textbf{Предмет исследования:}
эффективность методов оценки средней доходности и оценка доходностей соотвествующих портфелей.

\textbf{Методы исследования:}
методы теории вероятностей, математической статистики и временных рядов, методы регрессионного анализа, методы машинного обучения.

\textbf{Результаты работы:}
предложены методы оценки средних доходностей и их ковариаций между активами. На реальных данных исследованы доходности соотвествующих портфелей.
Выполнена программная реализаци алгоритмов по определению оптимальных портфелей и оценка их доходностей.

\textbf{Области применения:}
фондовые, валютные, криптовалютные биржи. Инвестиционные проекты, страхование.

\textbf{Структура магистерской диссертации:}
работа изложена на \pageref{LastPage} страницах, состоит из общей характеристики на 3 языках, введения, 
\totalchapters{} глав, заключения, списка использованных источников и приложения.
Содержит \totalfigures{} рисунков, \totaltables{} таблиц и 1 приложение.

% BELARUSIAN
\newpage
\addcontentsline{toc}{chapter}{АГУЛЬНАЯ ХАРАКТЫРЫСТЫКА РАБОТЫ}
\begin{center}
	\textbf{\large АГУЛЬНАЯ ХАРАКТЫРЫСТЫКА РАБОТЫ}
\end{center}

\textbf{Ключавыя словы:}
ПАРТФЕЛЬНАЯ ТЭОРЫЯ, IНВЕСТЫЦЫI, АКТЫВЫ, ВАЛЮТЫ, КРЫПТАВАЛЮТЫ, СЯРЭДНЕ-ДЫСПЕРСIЁННЫ АНАЛІЗ, ПРАГНАЗІРАВАННЕ ЧАСОВЫХ ШЭРАГАУ,
ДАХОДНАСЦЬ, АУТАРЭГРЭСIЯ, МАШЫННАЕ НАВУЧАННЕ, БIРЖА.

\textbf{Мэта работы:} 
даследаваць на рэальных дадзеных эфектыўнасць метадаў ацэнкі сярэдняй чаканай даходнасці ў партфельнай тэорыі Маркавiца.

\textbf{Аб’екта даследавання:} 
метады прагназавання сярэдняй даходнасці, партфельная тэорыя.

\textbf{Прадмет даследавання:}
эфектыўнасць метадаў ацэнкі сярэдняй даходнасці і ацэнкі даходнасці адпаведных партфеляў.

\textbf{Метады даследавання:}
метады тэорыі верагоднасцей, матэматычнай статыстыкі і часовых шрагау, метады рэгрэсійнага аналізу, метады машыннага навучання.

\textbf{Вынікі работы:}
прапанаваныя метады ацэнкі сярэдніх даходаў і іх каварыацый паміж актывамі. Па рэальных дадзеных даследаваны даходнасці адпаведных партфеляў.
Выканана праграмная рэалізацыя алгарытмаў па вызначэнні аптымальных партфеляў і ацэнцы іх даходаў.

\textbf{Вобласть ўжывання:}
фондавыя, валютныя, криптовалютные біржы. Інвестыцыйныя праекты, страхаванне.

\textbf{Структура магістэрскай дысертацыі:}
праца выкладзена на \pageref{LastPage} старонках, складаецца з агульных характарыстык на 3 мовах, увядзенні,
\totalchapters{} главы, заключэнні, спісы выкарыстаных крыніц і дадаткаў.
Змяшчае \totalfigures{} малюнкаў, \totaltables{} табліцу і 1 дадатак.

% ENGLISH
\newpage
\addcontentsline{toc}{chapter}{GENERAL DESCRIPTION OF WORK}
\begin{center}
	\textbf{\large GENERAL DESCRIPTION OF WORK}
\end{center}

\textbf{Keywords:}
PORTFOLIO THEORY, INVESTMENTS, ASSETS, CURRENCIES, CRYPTOCURRENCIES, MEAN-VARIANCE ANALYSIS, 
TIME SERIES FORECASTING, RETURN, AUTOREGRESSION, MACHINE LEARNING, STOCK EXCHANGE.

\textbf{The aim:}
to investigate the effectiveness of methods for estimating the average expected return in Markowitz's portfolio theory on real data.

\textbf{The object:}
methods for forecasting average returns, portfolio theory.

\textbf{Research methods:} 
methods of probability theory, mathematical statistics and time series, methods of regression analysis, methods of machine learning.

\textbf{The results:}
Methods for estimating average returns and their covariances between assets are proposed. 
The returns of the corresponding portfolios are studied using real data. 
A software implementation of algorithms for determining optimal portfolios and estimating their returns is completed.

\textbf{Application:}
stock, currency, cryptocurrency exchanges. Investment projects, insurance.

\textbf{Structure of a Master's Thesis:}
the work is presented on \pageref{LastPage} pages, consists of a general description in 3 languages, an introduction,
\totalchapters{} chapters, a conclusion, a list of references and an appendix.
Contains \totalfigures{} figures, \totaltables{} tables and 1 appendix.