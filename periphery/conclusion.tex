\newpage
\begin{center}
	\textbf{\large ЗАКЛЮЧЕНИЕ}
\end{center}
\refstepcounter{chapter}
\addcontentsline{toc}{chapter}{ЗАКЛЮЧЕНИЕ}

В работе была рассмотрена проблема формирования оптимального портфеля с точки зрения ожидаемой доходности и принимаемого риска.
Были предложены и протестированы модели ожидаемой средней доходности активов внутри подхода формирования оптимального портфеля по Марковицу.
На основании прогнозов этих моделей и исторической ковариации
между активами, формируется множество парето-оптмальных портфелей, соответсвующих заданному уровню риску.

Полученные портфели были протестированы на реальных данных за 2024 год. Результаты проверки стратегий:
\begin{enumerate}
	\item Активы имеют сильную положительную корреляцию
	\item С помощью диверсификации можно добиться снижения рисков
	\item Эмпирические фронтиры стратегий имеют выпуклую вверх форму, что согласуется с теорией
	\item Сформированные портфели оптимальнее инвестирования в отдельные активы
\end{enumerate}